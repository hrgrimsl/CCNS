\documentclass{article}
\usepackage{amsmath}
\usepackage{fullpage}
\usepackage{amsmath}
\title{Homework 6}
\author{Harper Grimsley}
\begin{document}
\maketitle
\section*{4.5}
\subsection*{2}
We note that
$$(1,1)=\int_{-1}^1 1dt = 2$$
so 
$$u^{(0)} = \frac{1}{\sqrt{2}}$$ 
We now find $v^{(1)}$.
$$\alpha_0 = (t, \frac{1}{\sqrt{2}}) = \int_{-1}^1 \frac{t}{\sqrt{2}}dt = 0$$
so 
$$v^{(1)}=t$$
and 
$$(t,t) = \int_{-1}^{1}{t^2}dt= \frac{2}{3}$$
so 
$$u^{(1)} = \sqrt{\frac{3}{2}}t$$
We now find $v^{(2)}$.
$$\alpha_0 = (t^2, \frac{1}{\sqrt{2}}) = \frac{1}{\sqrt{2}}\int_{-1}^1 t^2dt=\frac{\sqrt{2}}{3}$$
and
$$\alpha_1 = (t^2, \sqrt{\frac{3}{2}}t) = 0$$
since any multiple of $t^4$ is symmetric about $x=0$.  Finally,
$$(t^2-\frac{1}{3},t^2-\frac{1}{3})  = \frac{8}{45}$$
so 
$$u^{(2)}=\sqrt{\frac{45}{8}}({t^2}-\frac{1}{3})$$
\subsection*{4}
If $x$ is an eigenvector of $(\alpha_{ij})$ with eigenvalue 0, then for every row i of (a), we have 
$$\sum_j \alpha_{ij}x_j=0$$
or
$$\sum_ja^{i*}a^jx_j=0$$
Since $a^{i*}$ must be non-zero by linear independence,
$$\sum_ja^{j}x_j = 0$$
or since the $a^j$'s are simply the transposes of the rows:
$$\sum_ia^{iT}x_i= 0$$
or
$$(\sum_i\bar{x}_{i}\bar{a}^i)^* = 0$$
or
$$||\sum_i\bar{x}_i\bar{a}_i||^2=||\sum_i{x}_i{a}_i||^2 = 0$$
as required.
\section*{4.6}
\subsection*{1}
$$A = \begin{pmatrix}4&3/2\\3/2&2\end{pmatrix}$$
A has eigenvalues $3+\frac{\sqrt{13}}{2}$ and $3-\frac{\sqrt{13}}{2}$, and corresponding unit eigenvectors
$$u_1 = \frac{1}{26-4\sqrt{13}}(3,-2+\sqrt{13})^T, u_2 = \frac{1}{26+4\sqrt{13}}(3,-2-\sqrt{13})^T$$
These vectors should have respective lengths of 
$$ \frac{1}{\sqrt{3+\frac{\sqrt{13}}{2}}}, \frac{1}{\sqrt{3-\frac{\sqrt{13}}{2}}}$$
to become principle axes.
Thus, the principle axes are computed as
$$y^1 = \frac{1}{\sqrt{3+\frac{\sqrt{13}}{2}}(26-4\sqrt{13})}(3,-2+\sqrt{13})^T$$
$$y^2 = \frac{1}{\sqrt{3-\frac{\sqrt{13}}{2}}(26+4\sqrt{13})}(3,-2-\sqrt{13})^T$$
For the stated transform, 
U is the matrix with columns $u_1, u_2$ as given above, and 
$$\Lambda = \begin{pmatrix}3+\sqrt{13}/2&0\\0&3-\sqrt{13}/2\end{pmatrix}$$
Eq (19) would look like
$$1 = (3+\sqrt{13}/2)y_1^2 +(3-\sqrt{13}/2)y_2^2$$
where $y_1$ and $y_2$ refer to coordinates in the $\{y^1,y^2\}$ basis.
\section*{4.7}
\subsection*{3}
A does not obey this property.  For example, if $A = \begin{bmatrix}0&i\\0&0\end{bmatrix}$ with eigenvector $v = (1,0)^T$, the vector $u = (0,1)^T$ is orthogonal to v, but $Au = (i, 0)^T$ which is not orthogonal to v, so A does not map the space of vectors orthogonal to an eigenvector onto itself.
\subsection*{4}
Let U be the matrix whose columns are linearly independent eigenvectors of A, guaranteed to exist by the n distinct eigenvalues of A.  Let $\Lambda$ be the diagonal matrix of those eigenvalues in the same 'order' as in X.  (i.e. as in the canonical diagonalization.)  Then 
$$AU = U\Lambda$$
or
$$A = U\Lambda U^{-1}$$
or
$$A^* = U^{-1*}\bar{\Lambda}U^{*}$$
or
$$A^{*}U^{-1*} = U^{-1*}\bar{\Lambda}$$
We now show that U can be chosen to be unitary.  Let $u^1$ be the first eigenvector scaled by its length to have length 1.  Now the space of eigenvectors orthogonal to $u^1$ is invariant to A*, since if 
$$(u^1,x) = 0$$
then
$$(x,u^1) = 0$$
and
$$(u^1,A^*x) = (Au^1,x)=\lambda^1(u^1,x) = 0$$
Thus, we can choose an eigenvector $u^2$ which is orthogonal to $u^1$ and scale it by its length to have length 1.  We continue this until we have n unit eigenvectors which are orthogonal and construct U from them.  Since U is unitary, our earlier equation gives that
$$A^*U = U\bar{\Lambda}$$
such that the eigenvectors of $A^*$ are the same as those of A, but with their eigenvalues the complex conjugates of A's eigenvalues.  That is, $u^j = v^j$, so $(u^j,v^k)=(u^j,u^k)=\delta_{jk}$ as required. 
\section*{4.10}
\subsection*{1}
As the book points out, we simply need to find the unit eigenvector of R with $\lambda = 1$.  This turns out to be
$$u = \frac{1}{\sqrt{1+\arcsin^2(\cos\theta-x)}}\begin{pmatrix}1&\arcsin(\cos\theta-x)\end{pmatrix}$$ 
\subsection*{4}

\section*{Collaboration}
I collaborated with the usual group of chemists.
\end{document}
