\documentclass{article}
\usepackage{amsmath}
\usepackage{fullpage}
\usepackage{amsmath}
\title{Homework 6}
\author{Harper Grimsley}
\begin{document}
\maketitle
\section*{4.5}
\subsection*{2}
We note that
$$(1,1)=\int_{-1}^1 1dt = 2$$
so 
$$u^{(0)} = \frac{1}{\sqrt{2}}$$ 
We now find $v^{(1)}$.
$$\alpha_0 = (t, \frac{1}{\sqrt{2}}) = \int_{-1}^1 \frac{t}{\sqrt{2}}dt = 0$$
so 
$$v^{(1)}=t$$
and 
$$(t,t) = \int_{-1}^{1}{t^2}dt= \frac{2}{3}$$
so 
$$u^{(1)} = \sqrt{\frac{3}{2}}t$$
We now find $v^{(2)}$.
$$\alpha_0 = (t^2, \frac{1}{\sqrt{2}}) = \frac{1}{\sqrt{2}}\int_{-1}^1 t^2dt=\frac{\sqrt{2}}{3}$$
and
$$\alpha_1 = (t^2, \sqrt{\frac{3}{2}}t) = 0$$
since any multiple of $t^4$ is symmetric about $x=0$.  Finally,
$$(t^2,t^2) = \int_{-1}^1 t^4 dt = \frac{2}{5}$$
so 
$$u^{(2)}=\sqrt{\frac{5}{2}}({t^2}-\frac{1}{3})$$
\subsection*{4}
If $x$ is an eigenvector of $(\alpha_{ij})$ with eigenvalue 0, then for every row i of (a), we have 
$$\sum_j \alpha_{ij}x_j=0$$
It follows that
$$||\sum_j\alpha_{ij}x_j||^2=0$$
or
$$\sum_{ij}\alpha_{ij}^*\alpha_{ij}x^{j*}x^j=0$$
or
$$\sum_{ij}a^{j*}a^ia^{i*}a^jx^{j*}x^j=0$$
or
$$\sum_i{||a_i||^2}\sum_j a^{j*}a^jx^{j*}x^j=0$$
The sum of squared lengths of a the a's cannot be 0 if the a's are linearly independent and therefore non-zero, so it follows that
$$\sum_j a^{j*}a^jx^{j*}x^j=0$$
or
$$\sum_j ||a^j||^2||x^j||^2=0$$
This implies that $a^j$ or $x$ is 0, so it is a weaker corollary that 
$$||\sum_j ||ax||^2 = 0$$.  This implies that the only time AX=0 is when X = 0, so A is non-singular.

\end{document}
