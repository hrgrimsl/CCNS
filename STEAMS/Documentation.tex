\documentclass{article}
\usepackage{braket}
\usepackage{amsmath}
\usepackage{fullpage}
\usepackage{tikz}
\include{ccdiag}

\title{Documentation for Developers}
\author{Harper Grimsley}
\begin{document}
\maketitle
\section*{Tensor Conventions}
\begin{paragraph}{}
Both the gradient and trial vectors are considered in terms of dictionaries of "blocks" following spin-integration, with redundancy coming only from the antisymmetry condition.  Each four-tensor and two-tensor "block" B will have 
\begin{equation}\nonumber
\ket{\psi_{ij}^{ab}}\rightarrow B_{ijab}
\end{equation}
and
\begin{equation}\nonumber
\ket{\psi_{i}^{a}}\rightarrow B_{ia}
\end{equation}
respectively.  The overall structure of a vector will look like:
$$\begin{pmatrix}
S_{\alpha}\\
S_{\beta}\\
D_{\alpha\alpha}\\
D_{\alpha\beta}\\
D_{\beta\alpha}\\
D_{\beta\beta}\\
\end{pmatrix}$$
where each S is a two-tensor of single excitations, and each D is a four-tensor of double excitations.  To avoid ambiguity, double excitations will be mapped as:
\begin{equation}\nonumber
\ket{\psi_{\alpha\beta}^{\alpha\beta}}\rightarrow D_{\alpha\beta}
\end{equation}
The Hessian is composed of 36 blocks:
$$\hat{\eta} = \begin{pmatrix}
S_\alpha/S_\alpha&S_\alpha/S_\beta&S_\alpha/D_{\alpha\alpha}&S_\alpha/D_{\alpha\beta}&S_\alpha/D_{\beta\alpha}&S_\alpha/D_{\beta\beta}\\

S_\beta/S_\alpha&S_\beta/S_\beta&S_\beta/D_{\alpha\alpha}&S_\beta/D_{\alpha\beta}&S_\beta/D_{\beta\alpha}&S_\beta/D_{\beta\beta}\\

D_{\alpha\alpha}/S_\alpha&D_{\alpha\alpha}/S_\beta&D_{\alpha\alpha}/D_{\alpha\alpha}&D_{\alpha\alpha}/D_{\alpha\beta}&D_{\alpha\alpha}/D_{\beta\alpha}&D_{\alpha\alpha}/D_{\beta\beta}\\

D_{\alpha\beta}/S_\alpha&D_{\alpha\beta}/S_\beta&D_{\alpha\beta}/D_{\alpha\alpha}&D_{\alpha\beta}/D_{\alpha\beta}&D_{\alpha\beta}/D_{\beta\alpha}&D_{\alpha\beta}/D_{\beta\beta}\\

D_{\beta\alpha}/S_\alpha&D_{\beta\alpha}/S_\beta&D_{\beta\alpha}/D_{\alpha\alpha}&D_{\beta\alpha}/D_{\alpha\beta}&D_{\beta\alpha}/D_{\beta\alpha}&D_{\beta\alpha}/D_{\beta\beta}\\

D_{\beta\beta}/S_\alpha&D_{\beta\beta}/S_\beta&D_{\beta\beta}/D_{\alpha\alpha}&D_{\beta\beta}/D_{\alpha\beta}&D_{\beta\beta}/D_{\beta\alpha}&D_{\beta\beta}/D_{\beta\beta}\\
\end{pmatrix}$$
As an example, consider the $S_\alpha$ block of the action, $\vec{a}$, of $\hat{\eta}$ on a trial vector $\hat{\theta}$.
\begin{equation}\nonumber
S_{\alpha}^{\vec{a}}=\begin{pmatrix}
S_\alpha/S_\alpha&S_\alpha/S_\beta&S_\alpha/D_{\alpha\alpha}&S_\alpha/D_{\alpha\beta}&S_\alpha/D_{\beta\alpha}&S_\alpha/D_{\beta\beta}\\
\end{pmatrix}^{\hat{\eta}}
\begin{pmatrix}
S_{\alpha}\\
S_{\beta}\\
D_{\alpha\alpha}\\
D_{\alpha\beta}\\
D_{\beta\alpha}\\
D_{\beta\beta}\\
\end{pmatrix}^{\vec{\theta}}
\end{equation}
where the dot product operator is spin-integrated tensor contraction instead of multiplication.  This arithmetic is easily generalized to vector multiplications.  For the sake of explicitness, we consider the 
$$D_{\alpha\beta}/S_{\alpha}^{\hat{\eta}}\rightarrow S_{\alpha}^{\vec{\theta}}\implies D_{\alpha\beta}^{\vec{a}}$$ contraction, specifically the following contraction:\\\\

\begin{equation}\nonumber
\bdiag
    \dHtwo{1}{2}
    \dline[$\alpha$]{1}{1v1}
    \dline[$\alpha$]{1vd1}{1}
    \dline[$\beta\beta$]{2}{2v2}
    \dline[]{2v1}{2}
\ediag\hspace{10pt}\rightarrow\hspace{10pt}\bdiag
    \dT{1}{1}
    \dline[$\alpha\alpha$]{1}{1v2}
    \dline{1v1}{1}
\ediag\hspace{10pt}\implies\hspace{10pt}\bdiag

\dmoveH{2}

\dHtwo{h1}{h2}

\dT{1}{t}
\dline{tv1}{t}
\dline[ic]{t}{h1}
\dline[a]{h1}{h1v1}
\dline{h2}{h2v1}
\dline[bj]{h2v2}{h2}

\ediag
\end{equation}
Note that we place $\alpha$ q-particles on the left and $\beta$ q-particles on the right to be consistent with the fact that we want to recover part of the $D_{\alpha\beta}$ action from this contraction.  This is mostly pedagogical, as it makes it easier to keep track of things like permutations.  In general, this contraction looks like would contribute
$$\vec{a}_{ijcb} = \sum_c\mathcal{P}_{ij}\braket{ab||cj}\vec{\theta}_{ic}$$ 
However, we wish to spin-integrate this.  First, we note that the exchange contribution will be 0 because a and j are of opposite spins.  (As are b and c).  Additionally, we know that permuting i and j will lead to the requirement that a $\beta$ hole line is entering $T_1$.  Therefore, this term reduces to: 
$$\vec{a}_{ijcb} = \sum_c\braket{ab|cj}\vec{\theta}_{ic}$$ 
Note that for consistency, our vector elements are read from diagrams as in-in-out-out.
\end{paragraph}
\end{document}
